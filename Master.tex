\documentclass[12pt,english]{article}
\usepackage{color,soul}
\usepackage[T1]{fontenc}
\usepackage{inputenc}
\usepackage{babel}
\usepackage{url}
\usepackage{graphicx}
\usepackage[authoryear]{natbib}
\usepackage[unicode=true]{hyperref}
\usepackage[format=hang, justification=raggedright]{subfig}
\usepackage{array}
\usepackage{multirow}
\usepackage{lscape}
\usepackage{float}
\usepackage{longtable}
\usepackage{array}
\usepackage[noae]{Sweave}
\usepackage{setspace}
\doublespacing
\usepackage{geometry}
\geometry{verbose,tmargin=2cm,bmargin=2cm,lmargin=3cm,rmargin=3cm}
\usepackage{amsmath}
\usepackage[flushleft]{threeparttable}

\begin{document}
\input{Master-concordance}

\title{Income and Food Price Transmission\footnote{Revised version of MS 2015310 ``Income and Price Transmission''. Submitted to \emph{Agricultural Economics} for 2. round review. I thank two anonymous reviewers for their valuable comments.}}

\date{\vspace{-5ex}}

\maketitle

\begin{abstract}
This paper examines determinants of international-to-domestic food price transmission using aggregate monthly price data for more than 100 countries covering the ten year period 2005-15. Findings indicate that middle-income countries experience the highest rates of food price transmission in the long run. Other determinants of food price transmission include a country's cereal import, export and consumption shares, its trade openness, the quality of its infrastructure and government involvement in grain trade. A novel model is presented which is able to explain the economic logic behind the observed inverse-U shaped relationship between income and price transmission.

\end{abstract}

\textbf{Keywords:} price transmission, food crisis, food inflation, determinants, regression analysis, income\newpage

% !Rnw root = master.Rnw

\section{Introduction}
In response to the substantial increase in international food prices during the first decade of the new millennium, a large number of recent studies have quantified global and local poverty impacts of international price shocks.\footnote{E.g. \cite{Attanasio2013,Balagtas2014,Coxhead2012,Ferreira2013,Hoyos2011,Ivanic2012,Janvry2010,Mghenyi2011,Vu2011,Wodon2010}. In addition, a supplement to volume 39 of \emph{Agricultural Economics} is entirely devoted to this topic \citep[see][]{masters08}.} A major determinant of a country's vulnerability to such shocks is the responsiveness of its domestic food prices \citep[e.g.][]{Hoyos2011}. But what are the main determinants of the domestic price responsiveness to international shocks? Are they related to trade, consumption, geography or something else? As the title of the paper suggests, a country's income level seems to be an important determinant but other factors also play a role. In particular, countries which control their international trade through an agricultural state trading enterprise are found to experience lower food price transmission on average in the short run as well as in the long run.

The literature on domestic price impacts of international commodity price shocks is vast and the topic has experienced a revival in the wake of the 2007-08 food crisis \citep[see][]{Fackler2001, kouyate16}. However, since the majority of price transmission studies focus on a single country we know little about the determinants, other than distance and borders, of price transmission differences across countries. Moreover, most price transmission studies focus on individual commodity prices rather than food prices in general. This paper therefore contributes with a cross country analysis of price transmission determinants that is based on aggregate food price data, i.e. food price indexes, rather than prices of individual agricultural commodities.

Regarding the choice between individual or aggregate prices, each approach has its strengths and weaknesses \citep{headey10}. Commodity specific analyses have a stronger microeconomic foundation in the law of one price but results do not generalize to other food goods or even to the same food good traded in other domestic markets. Transmission estimates based on aggregate price data, on the other hand, give a more comprehensive picture and therefore better reflect domestic welfare costs of international food price fluctuations. A potential weakness though of the latter approach, is that a country's food price index might not reflect the composition of poor people's food baskets which is often based mainly on grain. Another weakness is that the composition of a country's food price index differs substantially from that of the international food price indexes and as well as from other countries' price indexes, which makes it more difficult to compare transmission rates across countries. These idiosyncrasies do not, however, preclude us from studying broad patterns in price transmission rates.

The empirical analysis is based on insights derived from a simple economic model that is more suitable for aggregate price data than the law of one price workhorse model. A key prediction is that a country's income level should have an independent effect on price transmission, in addition to factors related to trade openness, geography and so on. The model predicts a novel inverse-U shaped relationship between a country's income level and its price transmission rate. This income-price transmission relationship turns out to be a robust feature of the data. 

There are also other significant predictors of the domestic responsiveness to international food price shocks. Countries which import a large share of their grains and countries whose diets are mainly based on cereals are, surprisingly, found to have lower price transmission on average whereas the opposite holds for countries which export most of their grain. Countries that are more open in general in terms of their trade to GDP ratio also experience more price transmission whereas countries with an agricultural state trading enterprise experience lower price transmission. These country characteristics also significantly explain differences in domestic food inflation during the 2007-08 food crisis.

What are the policy implications of these findings? One obvious implication is that we may have to revise somewhat our perception of vulnerability and country focus in connection with international food crises. Donors and international aid organizations tend to a focus mostly on low-income food deficit countries but, according to the estimates in this paper, it seems that poor households living in middle-income countries are actually the most vulnerable to an international food price spike. Moreover, it does not seem to be the case that international price changes in general are passed on at a higher rate to countries that import most of their grains and this was not the case either during the 2007-08 food crisis.

The paper now proceeds as follows. Section \ref{sec:model} discusses the determinants of price transmission and lays out a model of a country's food price index. Section \ref{sec:analysis} presents the empirical analysis and, finally, section \ref{sec:conclusion} offers some concluding remarks.
% !Rnw root = master.Rnw

\section{Determinants of food price transmission\label{sec:model}}
What are the main determinants of a country's responsiveness to an international food price shock? I.e., which factors would we expect to influence the relationship between international and domestic food prices? To answer this question this section first reviews literature. Next, a simple model is presented of a country's food price level and its responsiveness to international shocks based on insights from the literature review. 

\subsection{Literature review}
There are only a few studies carrying out actual cross section analyses of price transmission determinants. There are, however, several studies discussing this informally, especially in relation to the 2007-08 food crisis. Arguments and main findings from a selection of these studies are summarized below.

\citet{Hoyos2011} point out that domestic food inflation during the food crisis varied a lot across countries.\footnote{Domestic food inflation is defined as the percentage change in a country's Food Price Index (FPI) which is a component of the Consumer Price Index (CPI).} Furthermore, domestic food inflation was lower on average than international food inflation. \citeauthor{Hoyos2011} offer 4 reasons why was is the case: (i) weak price transmission perhaps due to poor infrastructure, (ii) differences in the composition of food baskets and the fact that domestic food baskets contain nontradables, (iii) lack of competition in the domestic food markets and (iv) government interventions for example in the form of price controls.

\citet{Abbott2009} discusses the causes and consequences of the food crisis. Regarding the cross country differences in food inflation rates, he makes three observations: i) developing countries have more basic diets which involve a greater share of staples; ii) processing and distribution margins are typically lower than in developed countries, which means that they can shrink less as costs increase. Both factors, he argues, caused a stronger domestic food inflation response among developing countries than in developed countries to the high world prices in 2007-08. And, iii) a country's exchange rate regime has a large impact on the domestic price impact of an international shock; countries which peg their currency to the dollar are more exposed to international price fluctuations than countries whose currencies follow the euro.

\citet{Lee2013} quantify the importance of various ``external'' and ``internal'' variables on domestic food inflation based on data from 72 countries over the period 2000 - 2011. The set of external variables consists of current and lagged international food inflation, as well as a set of intra- and extra-regional food inflation averages. The internal variables represent domestic supply and demand factors as well as various other country characteristics. What they find is that  countries with large food import shares have lower food inflation whereas countries experiencing large increases in their food import shares have higher food inflation on average. The latter result could, however, reflect that food imports increase when domestic domestic food prices are on the rise, as a response to food inflation rather than a cause of it. Surprisingly, the authors also find a negative relationship between a country's per capita GDP growth rate (a demand variable) and its food inflation rate. The relationship between domestic food inflation and the per capita GDP level is found to be positive albeit only marginally significant in some specifications. A depreciation of the domestic currency/US\$ exchange rate increases domestic food inflation significantly across all specifications, as expected, whereas an increase in the money supply has no effect on food inflation. More politically stable countries tend to have lower food inflation whereas countries with a higher free trade index score are found to have higher domestic food inflation.

Two studies quantify the determinants of price transmission and market integration based on individual agricultural commodity prices rather than food price indexes. \citet{kouyate16} carry out a meta-regression analysis of the effect of distance on agricultural commodity price transmission for a large number of market pairs. Their results show that geographical distance and separation by borders reduces the likelihood of price cointegration and weakens the speed of speed of price transmission significantly. These findings are consistent with the well established stylized facts from the trade literature, that borders and distance reduces trade.

Finally, \citet{greb12} consider 8 variables covering geographical, infrastructural, institutional and commodity specific factors which are believed to influence agricultural price transmission. The authors regress their own price transmission estimates, as well as estimates obtained from the literature, on these country characteristics. What they find is that the covariates explain more of the variation in the estimated speed of adjustment parameters (the $\alpha$'s) than in the cointegration parameters (the $\beta$'s). Another major finding is that more of these covariates have significant and plausible partial effects on the $\alpha$'s than on the $\beta$'s. However, none of the determinants have a robust robust impact on price transmission.

\subsection{A model of a country's food price level}
This section presents a simple model of a country's food price level that is inspired by the arguments and findings discussed above, related to determinants of domestic food inflation and price transmission. The benefit of a working with a formal economic model is that it imposes some structure on the empirical analysis. In fact, despite its simplicity, it is possible to derive several predictions from the model regarding the determinants of price transmission.

A country's consumer price index (CPI) can be expressed as a weighted average
\begin{equation}
P =a P_{f}+(1-a)P_{n}\label{eq:cpi},
\end{equation}
where $P_{f}$ and $P_{n}$ denote the food and non-food price indexes, respectively, and $a$ is the food expenditure share. Time indexes are omitted in order to avoid notational clutter. The food price index (FPI), which is of primary interest, can be written
\begin{equation}
P_{f} =b P_{c}+(1-b)P_{m}\label{eq:fpi},
\end{equation}
where $P_{c}$ and $P_{m}$ denote the farm cost and marketing components of the FPI, respectively, and $b$ is the farm cost or commodity share.\footnote{By marketing costs I refer to all other costs than those of the raw agricultural commodities included in the final consumer food price such as those associated with distribution, storage and promotion.}  Finally, the commodity component of the domestic FPI can be decomposed as 
\begin{equation}
P_{c} =c P_{w}+(1-c)P_{d}\label{eq:compi},
\end{equation}
where $P_{w}$ and $P_{d}$ denote the tradable and purely domestic subcomponents of the commodity price index, respectively, and $c$ is the share of tradable food commodities in $P_{c}$. The defining difference between tradable and domestic food commodities is that the former have prices which are determined on the world market whereas the latter prices are determined by domestic supply and demand. 

The three share parameters, $a$, $b$ and $c$ represent weights on price indexes that are progressively embedded into each other. By substituting (\ref{eq:compi}) into (\ref{eq:fpi}) we get
\begin{equation}
P_{f} = b c P_{w}+R\label{eq:fpi1},
\end{equation}
where $R=b(1-c)P_{d}+(1-b)P_{m}$ is a term which does not depend on world market prices directly. With these definitions a first-order approximation to transmission of international agricultural commodity price changes to domestic consumer food prices can be written
\begin{equation}
\theta\equiv\frac{\partial P_{f}}{\partial P_{w}}=bc,\label{eq:pt}
\end{equation}
i.e., price transmission depends multiplicatively on the share of tradable commodities in the commodity price index, $c$, and the commodity cost share in the FPI, $b$. Expression (\ref{eq:pt}) is based on the assumption that there is no knock-on effect on domestic domestic prices when world market prices change, as would be appropriate in the short term. If we relax this assumption and allow domestic prices to respond to international price changes through substitution effects we would have to add the term $b(1-c)\frac{\mathrm{d}P_{d}}{\mathrm{d}P_{w}}$. For simplicity, I will ignore this second order effect in the analysis below.

By imposing some structure on the two cost shares, $b$ and $c$ we will be able to analyze determinants of a country's price transmission. In particular, we can write these shares as functions of certain country characteristics:
\begin{equation}
b=b(x_{b}), \quad c=c(x_{c}).
\end{equation}
But what are these characteristics? As I will argue below, a country's income level should feature prominently among them.

We can reasonably assume that the wealthier the country, the lower the commodity cost share in the FPI, c.f. \citet{Abbott2009}. Formally, if $y$ denotes income then $y\in x_{b}$ and $\frac{\partial b}{\partial y} <0$. Furthermore, if we believe that countries become more integrated with the world market as they develop, not only due to better infrastructure and more open markets, but also due to for example higher quality standards associated with the adoption of modern production and management practices, then it would be appropriate to also assume $y\in x_{c}$ and $\frac{\partial c}{\partial y} >0$.\footnote{We must also impose the following set of restrictions, $\lim_{y\rightarrow \infty} b(y)=0$, $\lim_{y\rightarrow \infty} c(y)=1$, $\lim_{y\rightarrow 0^{+}} b(y)=1$ and $\lim_{y\rightarrow 0~{+}} c(y)=0$, since otherwise these functions could not represent cost shares.} These assumptions in combination imply an inverse-U shaped relationship between price transmission, $\theta$, and income.

Income is not the only determinant of price transmission of course. A country's openness to trade, the quality of its infrastructure and the composition of its food consumption set are all factors which we would expect to influence the transmission of world market price fluctuations to domestic food markets. Economies that are more open in general, perhaps for historical reasons, should have a higher tradables share and therefore experience more price transmission. Similarly with countries that, due to climatic or geographical reasons, are forced to import most of their food supply. On the other hand, there are countries that are self-sufficient due to poor infrastructure, perhaps in combination with landlockedness, resulting in prohibitive transport costs. The tradables share in these countries should be very low and so should the price transmission rate. Consumption patterns are influenced by cultural as well as geographical factors. In particular, some countries consume more grains than others and the commodity component in the food price index should therefore be larger. Consequently, domestic aggregate food prices are more exposed international shocks in such countries. Finally, there are policies that are capable of affecting both shares. Trade policies in particular have a direct effect on the tradability of a good.

To sum up, there are reasons to expect a domestic FPI reaction to price fluctuations on the international agricultural commodity markets which depends non-monotonically on income. Specifically, the simple model sketched above predicts an inverse-U shaped relationship between income and price transmission. Factors related to trade openness, consumption, geography and the quality of infrastructure also, arguably, affect price transmission. 
% !Rnw root = master.Rnw

\section{Empirical analysis\label{sec:analysis}}
This section assesses the impact of the determinants discussed above on international-to-domestic aggregate food price transmission. Section \ref{sec:data} introduces the data and provides some context. Results from the statistical analysis are presented in section \ref{sec:analysis1} and \ref{sec:analysis2}.

\subsection{Data\label{sec:data}}



\subsubsection{Domestic and international food prices}
Data on domestic prices are sourced from the International Labor Organization (ILO).\footnote{See \citet{ilo16}} ILO maintains a comprehensive database of domestic food price indexes (FPIs) and consumer price indexes (CPIs) that is used to monitor the cost of living and real wages. There are not many studies which draw on this database, however, despite a broad country coverage and long monthly price series. A major issue with this data is that historical series are not recalculated whenever a new base year is introduced. The same applies when other changes are made such as the inclusion or exclusion of certain items such as beverages or tobacco from the index. This means that many of the series have discrete level shifts associated with these changes. In some cases one series replace another after a period of overlap. Finally, many of the series contain missing values. To ``tidy up'' the raw data I carried out the following steps First, I restricted the sample to the period 2005:1-2014:12. Secondly, in cases where multiple complete series were available from the same country, a single one was selected. Thirdly, series with missing values were discarded. Fourthly, and most critical for the analysis below, level shifts associated with base year changes etc. were identified and removed.\footnote{Concretely I added the change in the break point from the previous value to the break point and all points succeeding it. The value of the adjusted series at the break point is thus the same as the one preceding it.} Finally, all the series were normalized with respect to a common base period (Jan 2007). In total 118 countries have complete price series for the entire ten year period considered whereas, for example, there are 145 countries with complete price series covering the 2005-11 subperiod. There is therefore a trade-off between the number of complete price series and the length of the series.

An important detail to know about this data is that many of the low income countries' price indexes derive from prices collected from a few major markets in the capital. Furthermore, weights and selected items are derived from household expenditure surveys which often target certain groups such as the urban middle class. Differences in consumption patterns around the world of course translate into differences in weights and in the composition of the food baskets themselves. This is probably the largest drawback of using index data in a cross section analysis of price transmission. That being said, even domestic prices of individual food commodities such as wheat or rice would not refer to the same good due to quality and varietal differences, so it is not the case that focusing on a few selected food items consumed by all countries would necessarily solve the problem.

International food prices are represented by the FAO food price index (FAO FPI) and in some cases its ``real'' version, namely the nominal FAO FPI deflated by the Manufactures Unit Value (MUV) index published by the World Bank.\footnote{see \citet{fao16} and \citet{worldbank16a}.}

\subsubsection{Domestic food inflation during the food crises}

There has been two recent international price spikes; one in 2007-8 and the other in 2010-11. The former spike especially is often referred to as ``the food crisis'' in the literature \citep[e.g.][]{Headey2008, Headey2013, timmer10, Ivanic2012, Martin2011}. Between January 2007 and June 2008 the FAO FPI rose by 91 points to index 226, corresponding to a 68 percent relative price increase. From the first peak in mid-2008 and to the trough in February 2009 the FAO FPI fell by 82 points before rising back up to index 240 in February 2011. 

The boxplot in figure \ref{fig:infl} depicts the empirical distribution of the sample countries' headline (CPI) inflation, their food (FPI) inflation as well as the change in their FPI relative to the CPI, termed real FPI (RFPI) growth, over the two time periods 2007:1-2008:6 and 2010:1-2010:12.

\begin{center}
[Figure \ref{fig:infl} around here] \\
\par\end{center}

The median country experienced an annualized rate of food and headline inflation of 9.7 and 7.1 percent, respectively, in 2007-08. Nominal food prices are not only driven by exogenous world market shocks; they follow a stochastic trend together with the non-food prices, representing underlying general (core) inflation. Dividing with the CPI is a simple way of purging the FPI of underlying domestic inflation. RFPI growth, which is approximately the difference between the FPI and CPI inflation rate, was 2.5 percent for the median country in 2007-08.\footnote{A problem with the real FPI is that domestic non-food prices are also affected by international commodity shocks. Low or even negative growth in real domestic food prices in these two periods could therefore reflect a response to the concurrent surge in non-food commodities, oil in particular, rather than a puzzling negative response to an international food price shock.} Domestic price increases were generally lower in 2010-11. The median country experienced food and headline inflation rates 3.9 and 3.5 percent and a modest RFPI growth of 0.8 percent. There is, however, a substantial variation in the inflation rates in both periods. 

The three remaining panels of figure \ref{fig:infl} contain alternative representations the same data. There appears to be some systematic variation; as the (mostly) positively sloped lowess lines indicate, a country hit by a large nominal and real FPI shock in 2007-8 was more likely to be hit by a relatively large shock again in 2010-11 and vice versa. This suggests that there are indeed characteristics which influences a country's response to a world market shock, an assertion that will be explored further in section \ref{sec:analysis1} and \ref{sec:analysis2}.

\subsubsection{Price transmission determinants}
Table \ref{tab:cs_data} in the appendix lists the variables that, based on the discussion in section \ref{sec:model}, are considered potentially important determinants of price transmission. A country's PPP adjusted per capita GDP is considered a proxy for its income level. Other covariates considered are a country's percentage dietary energy supply derived from cereals, roots and tubers, its cereal import- and export shares, a dummy indicating whether the country is a landlocked developing country (LLDC), its trade openness defined as the sum of its imports and exports relative to its GDP and a dummy indicating whether there is a grain state trading enterprise (STE) in the country. In addition there are two indexes included representing the regulatory performance of the country in question and the quality of its infrastructure. These two indexes are abbreviated DTF (Distance to Frontier) and LPI (Logistics Performance Index), respectively.\footnote{The covariates are similar to those of \citet{greb12}.}  

\subsection{Determinants of domestic food price changes during the food crisis\label{sec:analysis1}}
Table \ref{tab:07-08-fc-regressions} considers the cross sectional variation in domestic food price changes during the 2007-8 food crises. Each of the columns refer to a regression of FPI inflation or real FPI growth on the explanatory variables discussed above as well as the percentage change in a country's domestic currency/US dollar exchange rate during the food crisis. All variables except for GDP and the dummies have been standardized in order to facilitate comparisons of effect sizes. 

\begin{center}
[Table \ref{tab:07-08-fc-regressions} around here]
\par\end{center}

Columns (1) and (3) are based on regression models that include the full set of explanatory variables. Income has a significant inverse-U shaped effect as predicted. The other significant covariates are a country's cereal import and export shares, its trade openness, the STE dummy and its LPI. Grain exporters experienced more food inflation on average whereas importers experienced less. The coefficient to the STE dummy is negative and large in magnitude suggesting that countries with grain state trading enterprises were able to control domestic food prices better than other countries. Two of the significant variables have unexpected signs, namely a country's cereal import share and its quality of infrastructure and trade as summarized by its LPI. It is not clear why large grain importers experienced lower food inflation during the food crisis. It could be though, that large food importers intervene more in their food markets than other countries. This interpretation is consistent with the finding by \citet{Lee2013} that large food importers have lower food inflation in general but more research is needed before we can say anything conclusive about this. Better infrastructure should imply more integrated markets and stronger price transmission but countries with a high LPI score experienced lower food inflation on average during the food crisis. The effect on real FPI growth is insignificant though, which would suggest that these countries also experienced lower non-food inflation. Countries with depreciating currencies experienced significantly stronger FPI inflation as expected but the effect on real FPI growth is insignificant. The latter finding is not surprising given that exchange rate changes affect domestic food- as well as non-food prices.

A concern is that some of the variables are highly correlated which leads to large standard errors and insignificant effects. Therefore, the subset of covariates included in columns (2) and (4) are those which minimizes the Akaike information criterion (AIC) as determined by stepwise regression.\footnote{This was carried out using the \emph{MASS} package in \textsf{R} \citep{venables02}} Stepwise regressions have been criticized for leading to models where the remaining terms appear to be more important than they actually are. However, as can be seen, no substantial changes result from this exercise. 

What effect did the government interventions have on domestic food inflation during the food crisis? Table \ref{tab:int-regressions} in the appendix reports results from a set of regressions similar to those in table \ref{tab:07-08-fc-regressions} except that a set of dummies are included indicating whether or not the government intervened in various ways during the food crisis.\footnote{These dummies are based on table 1 in \citet{Demeke2009}. ``Export'' = 1 if the country Restricted or banned export, ``Price'' = 1 if the country administered price control or restricted private trade, ``Stock'' = 1 if the country released stocks (public or imported) at subsidized prices'', ``Tariff'' = 1 if the country reduced tariffs or customs fees on imports and ``Tax'' = 1 if the country suspended or reduced VAT or other taxes, all as of 1. December 2008.\label{int_note}} Some of the coefficients to these intervention dummies are substantial but they are all statistically insignificant. Table \ref{tab:1011-regressions} in the appendix contains results related to the 2010-11 international price spike. Again, estimated effects are similar to those in  table \ref{tab:07-08-fc-regressions} but most are statistically insignificant. In fact,  the cereal consumption share is the only robustly significant predictor of domestic food inflation in 2010-11.

\subsection{Determinants of price transmission \label{sec:analysis2}}
Although the results in section \ref{sec:analysis1} provide interesting insights into the cross sectional variation in domestic price changes during the two recent international food price spikes they have at least two shortcomings. First, none of the estimated relationships represent price transmission as such as there may be the other and perhaps more important causes of the observed domestic price changes. Secondly, the estimates are based on a specific short time period, which means that we ignore much the information contained in the individual price series and it also raises questions about the reliability of the results. To accommodate these shortcomings I approach the issue from two different angles. First, in line with the small existing literature on global food price transmission patterns, I estimate a set of long run price transmission rates based on country specific time series models. These estimates are then subsequently regressed on the same country characteristics as in section \ref{sec:analysis1}. Secondly, I estimate the effects of the predictor variables on short run price transmission directly in a panel model.

\subsubsection{Meta analysis of estimated long run price transmission rates\label{sec:meta-analysis}}
This section carries out a meta-regression analysis to see whether a country's estimated long run price transmission rate varies systematically with the proposed determinants of price transmission.

Following the literature, monthly domestic food price changes are regressed on lagged domestic food price changes as well as on contemporaneous and lagged international food price changes and exchange rates \citep[see][]{Ianchovichina2014, IMF2011, Kalkuhl2014}. That is, for each country $i$ an autoregressive distributive lag (ADL) models is fitted
\begin{equation}
\pi_{i,t}^{dom}=\delta_{i,0}+\sum_{s=1}^{p}\rho_{i,s}\pi_{i,t-s}^{dom}+\sum_{s=0}^{q}(\lambda_{i,s}\pi_{t-s}^{int}+\phi_{i,s}er_{t-s})+\epsilon_{i,t}\label{eq:time_series_model}
\end{equation}
where $\pi_{i,t}^{dom}$ represents country $i$'s monthly FPI inflation or RFPI growth at time $t$ (both approximated by the first differences to the log of its FPI or RFPI), $\pi_{t}^{int}$ denotes the FAO FPI or, in case of the RFPI growth model, the FAO RFPI at time $t$ and $er_{i,t}$ is the country $i$'s domestic currency/US\$ exchange rate, both in log differences.

The ADL model is appropriate when the causal impact is unidirectional. This could be the case when the domestic price is from a small country relative to the world market but, more often, this assumption is invoked when the domestic price variable is a Food Price Index (FPI), i.e. the food component of the Consumer Price Index (CPI). 

Prior to estimation I tested each of the series for unit roots and cointegration with international prices.\footnote{I tested for stationarity with the KPSS test \citep{kwiatkowski92} and for unit roots with the augmented Dickey-Fuller test \citep{dickey79}. I used the Phillips and Ouliaris ``Pz'' test \citep{phillips90} to test for cointegration. These tests are implemented in \textsf{R} via the packages \emph{tseries} \citep{trapletti15} and \emph{urca} \citep{pfaff08}, respectively.} Results indicated that the domestic FPI and real FPI series, are nonstationary in levels but stationary in first differences. Domestic price levels were also found not to be cointegrated with world prices. Consequently, each model was estimated in growth rates rather than levels in order to avoid spurious regressions.

Next, for each country, I calculated the long run multiplier (LRM) associated with a change in the the FAO food price index given by
\begin{equation}
\theta_{i}=\frac{\sum_{s=0}^{q}\lambda_{i,s}}{1-\sum_{s=1}^{p}\rho_{i,s}}.
\end{equation}


The LRM is a common measure of price transmission. \cite{IMF2011}, for example, contains an analysis of food price price transmission based on a dataset of monthly FPI inflation from 31 advanced economies and 47 emerging and developing economies over the period 2000 - 2011. Their median estimated LRM pass-through of a 1 percent increase in international food inflation is 0.18 percent in developed countries and 0.34 percent in emerging and developing countries. Similarly, \citet{Ianchovichina2014} report an average estimated LRM for the eighteen Middle East and North Africa (MENA) countries of 0.25 and the majority of the countries are reported to have  food price pass-through LRMs in the order of 0.2-0.4. Both studies use 12 lags in their country specific ADL models, i.e. they assume $p=q=12$.\footnote{\cite{IMF2011} do not include exchange rates in their regressions.} My own estimates based on a sample of 118 countries covering the period 2005:1 to 2014:12 are very similar. The average and median FPI LRM from a model with 12 lags of all variables are 0.32 and 0.22, respectively. That being said, it is not obvious that so many lags should be included in the models. When lag length selection is based on the Akaike Information Criterion (AIC), for example, the optimal number lags in most of the models turn out to be $p=q=1$.\footnote{For each country, I calculated the AIC for each of the $12^2$ combinations of $(p,q)\, \in (1,..,12)$. Then I selected the $(p,q)$ combination with the lowest AIC.} But when a common lag length of 1 is used for all countries, average and median FPI transmission LRMs are much lower, namely 0.07 and 0.03, respectively, whereas the average of the standard errors of the LRM estimates very similar.\footnote{I also tested for residual autocorrelation with the Durbin-Watson test. There was no evidence of residual autocorrelation in either case. Standard errors of the LRMs are based on the so called Bewley transformation of the ADL model instrumented with lags of the dependent variable.} As a compromise between the long lag lengths used in the literature which might cause problems with overfitting and the short optimal lag lengths implied by the AIC criterion which leads to LRM estimates that are lower than commonly found in the literature, the country specific models are based on common lag lengths of $p=q=6$. This choice leads to average long run FPI and RFPI transmission rates of 0.14 and 0.03, respectively. One interpretation of the very low RFPI LRM average is that shocks to a country's FPI is passed on almost fully to its non-FPI in the long run, a phenomenon known as propagation in the literature \citep[e.g.][]{pedersen10}.

The long run transmission estimates, $\hat{\theta}$, are regressed, on the same explanatory variables as in table \ref{tab:07-08-fc-regressions}. However, since the dependent variable is an estimate rather than an observed value and the model now represents an estimated dependent variable (EDV) regression \citep[see][]{hanushek74, lewis05}. A common approach in this situation is to use Weighted Least Squares (WLS) estimation with weights $w_{i}$ that are inversely proportional to the standard error of $\theta_{i}$ \citep{saxonhouse76}. This way more precise estimates are given a higher weight in the regression. The main reason though for using WLS rather than OLS is that the latter may suffer from heteroskedasticity which leads to inefficient coefficient estimates inconsistent standard errors. The inconsistent OLS standard errors can, however, easily be corrected with a robust standard error estimator. \citet{lewis05} uses formal reasoning backed up by a Monte Carlo analysis to argue that OLS with robust standard errors is likely to produce more reliable estimates than WLS based on inverse standard error weights, and that the latter is likely to produce downward biased standard errors.

Table \ref{tab:fpi-pt-regressions} presents estimates of FPI transmission determinants based on OLS [columns (1)-(2)] as well as WLS [columns (3)-(4)], both with robust standard errors. Again, all variables except per capita GDP and the dummies have been standardized and, as in table \ref{tab:fpi-pt-regressions}, column (1) and (3) represent regressions that are based on the full set of covariates whereas columns (2) and (4) represent regressions with the lowest AIC. 

\begin{center}
[Table \ref{tab:fpi-pt-regressions} around here]
\par\end{center}

First, as can be seen, results are robust to the choice of estimator. Secondly, although the results in table \ref{tab:fpi-pt-regressions} relate to determinants of FPI transmission rather than FPI inflation as in table \ref{tab:07-08-fc-regressions}, the estimated coefficients and significance patterns are similar across the two sets of regressions. A large grain consumption and grain import share is associated with lower price transmission and so is having a grain state trading enterprise. Large grain exporters, on the other hand, experience larger price transmission on average and so do more open countries. The signs of some of these coefficients could reflect systematic differences in the degree of government intervention in the domestic food market as discussed in section \ref{sec:analysis1}. Income again has a highly significant inverse-U shaped affect of FPI transmission and, oddly, countries with better infrastructure are once more found to experience lower price transmission on average. A possible explanation for the latter effect could be, that the LPI is simply another proxy for income.

Table \ref{tab:rfpi-pt-regressions} in the appendix contain regression results based on estimated long run RFPI transmission rates from country specific time series models with 6 lags. The signs of the coefficients are similar to those in table \ref{tab:fpi-pt-regressions} but most effects are statistically insignificant. Similarly, table \ref{tab:fpi-pt1-regressions} and \ref{tab:rfpi-pt1-regressions} in the appendix contain results based on FPI and real FPI transmission rates calculated from time series models with a single lag. These also point to only a few significant predictors of price transmission. A plausible reason is that the LRM estimates based on models with a single lag are relatively more noisy and therefore more difficult to explain. 

\subsubsection{A panel model of short run price transmission determinants}
The regression results in table \ref{tab:07-08-fc-regressions} and \ref{tab:fpi-pt-regressions} are based on a small number of observations which decreases the precision of the estimates. Furthermore, the estimates are potentially subject to unobserved heterogeneity bias. The last set of regressions therefore exploits the panel structure of the dataset to control for time invariant unobserved heterogeneity which potentially confounds the estimated relationships. Specifically the following model is fitted on monthly price data covering the full 2005-15 period 
\begin{equation}
\pi_{it}^{dom}=\beta_{0}+
\sum_{s=1}^{6}\rho_{s}\pi_{it-s}^{dom}+
\sum_{s=0}^{1}(\lambda_{s}\pi_{t-s}^{int}+\phi_{s}er_{it-s})+
\beta_{1}\pi_{t-1}^{int}\cdot x_{i} +
u_{it}\label{eq:panel_model},
\end{equation}
where $x_{i}$ is the vector of country characteristics. 

It is well known that dynamic panel models such as equation (\ref{eq:panel_model}) suffers from (Nickell) bias due to correlation between the lagged dependent variable and the error term. However, as demonstrated by \citet{nickell81}, the bias of the within estimator decreases at a rate $1/T$, i.e. endogeneity is mainly a problem in short panel datasets with only a few time periods $T$. In the present case where $T>100$, the bias should be negligible.

Estimation results based on the within estimator are shown in table \ref{tab:panel-regressions}.\footnote{Note that coefficients on the lagged dependent variable are omitted.}

\begin{center}
[Table \ref{tab:panel-regressions} around here]
\par\end{center}

Regarding the specification of the model, the main effect of $\pi_{t-s}^{int}$ on $\pi_{it}^{dom}$ was found to be at lag $s=1$ and this is also the reason why the interactions terms $\pi_{t-1}^{int}\cdot x_{i} $ refer to the one period lagged effect. Consequently, the set of coefficients to the interaction terms represent factors affecting short run price transmission.\footnote{Results were found to be robust to the exact number of included lags of $\pi_{it}^{dom}$.}

As can be seen, the signs of the estimated coefficients on the interaction terms are mostly consistent with the estimates in table \ref{tab:07-08-fc-regressions} and \ref{tab:fpi-pt-regressions}. Income has a significant inverse-U shaped effect on short run FPI transmission whereas the grain consumption ratio and landlockedness have significant effects, positive and negative, respectively, on short run RFPI transmission. Finally, countries with grain state trading enterprises are found to have significantly lower short run (nominal as well as real) food price transmission.

